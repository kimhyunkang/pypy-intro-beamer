\documentclass[slidestop]{beamer}
\usepackage{kotex}
\usepackage{hyperref}
\usepackage[accumulated]{beamerseminar}
\usepackage{beamertexpower}
\usepackage{beamerthemeGoettingen}
\usepackage{beamercolorthemebeaver}

\usepackage{graphicx}

\addtobeamertemplate{block begin}{}{\setlength{\parskip}{35pt plus 1pt minus 1pt}}

\title{PyPy - A Python Interpreter written in Python}
\author[]{모비젠 플랫폼연구팀 김현강\\\code{kimhyunkang@mobigen.com}}
\date{}

\begin{document}

%% 표지
\begin{slide}

\maketitle

\end{slide}

\section{PyPy - 소개}
\frame{
\begin{slide}
\frametitle{PyPy}

  \includegraphics[width=0.8\textwidth]{Pypy_logo} \\
  \url{http://pypy.org}

  \begin{itemize}
  \item Python으로 작성한 Python 구현체
  \item CPython(표준 인터프리터)은 인터프리터가 모두 C로 작성되어 있습니다
  \item 반면 PyPy는 전부 Python으로 작성되어 있습니다
  \end{itemize}

\end{slide}
}

\frame{
\begin{slide}
\frametitle{PyPy의 역사}

  \begin{itemize}
  \item psyco의 개발자 Armin Rigo가 개발하였습니다
  \item 2007년 1.0 릴리즈가 발표되었습니다
  \item 현재 최신 릴리즈는 1.9이며, 2.0 베타가 발표되어 테스트 중에 있습니다
  \item Python 2.7.2 문법을 모두 지원합니다
    \begin{itemize}
    \item 표준 라이브러리도 모두 지원합니다
    \item Python 3은 아직 지원하지 않습니다
    \end{itemize}
  \item (Single-thread 환경에서) 평균적으로 가장 빠른 파이썬 구현체입니다
    \begin{itemize}
    \item Multi-thread 환경에서는 GIL이 없는 Jython이 더 빠를때도 있습니다
    \end{itemize}
  \end{itemize}

\end{slide}
}

\section{Bootstrapping}
\frame{
\begin{slide}
\frametitle{Bootstrapping}

  대상 언어로 대상 언어의 컴파일러/인터프리터를 짜는 테크닉을 bootstrapping이라고 합니다

  예
  \begin{itemize}
  \item GCC (C)
  \item Microsoft Visual C++
  \item Sun JDK
  \item Perl 6
  \item GHC (Haskell)
  \end{itemize}
\end{slide}
}

\frame{
\begin{slide}
\frametitle{Bootstrapping}
  Bootstrapping 컴파일러의 장점
  \begin{itemize}
  \item 컴파일러를 잘 만들기 위해 다른 언어를 잘 알 필요가 없습니다
  \item 다른 아키텍처로 포팅하기가 쉽습니다
  \item 컴파일러가 최적화된 코드를 만들면 런타임뿐만 아니라 컴파일 속도가 함께 빨라집니다
  \item 일반적으로 더 낮은 레벨의 언어로 컴파일러를 작성하게 되므로 컴파일러는 복잡해지기 쉽지만, bootstrapping 컴파일러는 대상 언어와 구현 언어가 같은 언어이므로 더 복잡한 기능을 쉽게 구현할 수 있습니다
  \end{itemize}
\end{slide}
}

\frame{
\begin{slide}
\frametitle{PyPy의 장점}
  빠릅니다
  \begin{itemize}
  \item 프로그램의 종류에 따라 다르지만, 순수 파이썬으로 작성된 프로그램의 경우 CPython보다 3배 이상 빠릅니다
  \item \url{http://speed.pypy.org/}
  \end{itemize}

  메모리를 더 잘 관리합니다
  \begin{itemize}
  \item 자료구조에 따라 메모리를 50\% 가량 절약할 수 있습니다
  \item CPython보다 garbage collector 성능이 더 좋습니다
  \end{itemize}

  CPython과의 호환성을 철저히 관리합니다
  \begin{itemize}
  \item 홈페이지에 명시된 몇가지 코너케이스를 제외하면, CPython에서 돌아가는 코드는 PyPy에서도 돌아갑니다
  \end{itemize}
\end{slide}
}

\section{PyPy 성능의 비밀}
\frame{
\begin{slide}
\frametitle{PyPy 성능의 비밀}
  파이썬으로 짠 인터프리터가 어떻게 CPython보다 빠를 수 있나요?
  \begin{itemize}
  \item PyPy를 CPython으로 돌릴 수도 있습니다
    \begin{itemize}
    \item 당연하게도 느립니다 (CPython의 200배 정도)
    \end{itemize}
  \item PyPy는 Python의 부분집합인 RPython으로 만들어져 있습니다
  \end{itemize}
\end{slide}
}

\subsection{RPython}
\frame{
\begin{slide}
\frametitle{PyPy가 빠른 이유 1 - RPython}
  PyPy 인터프리터의 파이썬 소스코드는 RPython이라는 컴파일러를 통해 C로 번역됩니다
  \begin{itemize}
  \item 컴파일 시 각 변수의 타입을 분석하여 그에 해당하는 C 타입으로 번역됩니다
  \item 인터프리터의 소스코드는 Python이지만, 실제로는 C 언어를 거쳐 다시 어셈블리로 번역되므로 인터프리터의 실행속도는 빠릅니다
  \item C외에 JVM이나 .net 바이트코드로 번역할 수도 있습니다
  \item 이 경우 JVM이나 .net에서 Python을 실행할 수 있습니다
  \end{itemize}
\end{slide}
}

\subsection{JIT Compilation}
\frame{
\begin{slide}
\frametitle{PyPy가 빠른 이유 2 - JIT Compilation}
  CPython은 코드 줄마다 여러가지 체크를 수행합니다 
  \begin{itemize}
  \item 인터프리터가 \textbf{a+b} 를 실행할 때 일어나는 일
  \item 오브젝트 \textbf{a}가 \textbf{\_\_add\_\_} 연산자를 갖고 있는지 검사합니다
  \item 없을 경우 \textbf{b}가 \textbf{\_\_radd\_\_} 연산자를 갖고 있는지 검사합니다
  \item 둘 중에 하나가 있다면, 두 값의 타입을 검사하여 coercion을 수행합니다
  \item 실제 덧셈이 수행됩니다
  \end{itemize}
\end{slide}
}

\frame{
\begin{slide}
\frametitle{PyPy가 빠른 이유 2 - JIT Compilation}
  PyPy는 파이썬 코드를 실행할 때 일부 코드를 기계어로 실시간 번역합니다
  \begin{itemize}
  \item 이런 테크닉을 Just-in-time 컴파일이라고 하며, Java VM이 빠른 이유이기도 합니다
  \item 모든 코드를 기계어로 번역하는 것은 실행시간보다 컴파일 시간이 오래 걸리므로 낭비입니다
  \item 루프 내의 코드나 자주 호출되는 함수가 나타나면 기계어로 번역합니다
  \end{itemize}
\end{slide}
}

\subsection{Garbage Collector}
\frame{
\begin{slide}
\frametitle{PyPy가 빠른 이유 3 - Garbage Collector}
  CPython은 reference-count로 메모리를 관리합니다
  \begin{itemize}
  \item 모든 CPython 변수에는 몇개의 변수가 자기 자신을 가리키고 있는지 추적하는 값이 붙어 있으며, 이 숫자가 0이 될때 메모리에서 해제합니다
  \item 따라서, 모든 변수가 메모리를 4바이트씩 더 사용합니다
  \item 변수가 서로를 순환 참조할 경우 메모리가 해제되지 않습니다
    \begin{itemize}
    \item 이를 막기 위해 cycle detector를 따로 실행합니다
    \end{itemize}
  \item 또한, 변수 해제시마다 연결된 모든 메모리에 접근하여 count를 감소시켜야 하므로, cache 성능이 떨어집니다
  \end{itemize}
\end{slide}
}

\frame{
\begin{slide}
\frametitle{PyPy가 빠른 이유 3 - Garbage Collector}
  반면 PyPy는 주기적으로 한 변수가 다른 변수를 가리키는 포인터를 모두 추적한 후 쓰이지 않는 메모리를 모아 해제합니다
  \begin{itemize}
  \item 오랫동안 해제되지 않는 변수들과 자주 할당되고 해제되는 변수를 구분하여 서로 다른 블록에 할당합니다
  \item 변수들을 수명 주기별로 모아 같은 블록으로 관리하며, 해제할때는 큰 블록을 한번에 OS에 반환합니다
  \end{itemize}

  잘 최적화된 Generational GC는 대부분의 경우 Reference-Counting보다 효율적이라는 것이 여러 runtime(특히 JRE)에서 증명되어 있습니다
\end{slide}
}

\section{PyPy의 약점}
\subsection{C 모듈}
\frame{
\begin{slide}
\frametitle{PyPy의 약점 1 - 일부 C 모듈을 지원하지 않음}
  많은 Python 모듈이 CPython API를 이용해 작성되어 있습니다.
  \begin{itemize}
  \item PyPy는 내부 구현이 CPython과 다르므로, 이를 이용해 작성된 모듈은 PyPy에서는 사용할 수 없습니다.
  \item PyPy 2.0에서는 CPython 호환성 기능을 베타테스트중입니다. 그러나 완전히 호환되는 단계는 아니라고 합니다
  \item Python 2.7 표준 라이브러리는 모두 지원합니다
  \end{itemize}

\end{slide}
}

\subsection{JIT}
\frame{
\begin{slide}
\frametitle{PyPy의 약점 2 - JIT가 최적화하지 못하는 코드}
  PyPy의 성능은 대부분 JIT에서 오며, JIT가 최적화하지 못하는 코드의 경우 CPython보다 느립니다
  \begin{itemize}
  \item 모듈 코드의 대부분이 C에서 동작할 경우, CPython이 더 빠릅니다
    \begin{itemize}
    \item 예: \textbf{sqlite3}, \textbf{cProfile}, \textbf{cPickle}
    \end{itemize}
  \item 반복 사용되는 코드가 거의 없을 경우
    \begin{itemize}
    \item JIT는 여러번 사용되는 코드를 최적화하며, 이에 걸리는 시간을 JIT warm-up이라고 합니다
    \item 반복 사용되는 코드가 없으면 JIT가 warm-up할 시간이 없어 빨라지기 힘듭니다
    \item 0.2초 이하의 짧은 코드의 경우 CPython이 대체로 빠르다고 합니다
    \item 대부분의 코드가 한번만 실행되는 경우 (예: 단위 테스트) CPython이 빠릅니다
    \end{itemize}
  \end{itemize}
\end{slide}
}

\frame{
\begin{slide}
\frametitle{PyPy의 약점 2 - JIT가 최적화하지 못하는 코드}
  언어 구조를 dynamic하게 변경하는 코드는 기계어로 번역하기 힘들어 JIT가 최적화하지 못합니다
  \begin{itemize}
    \item \textbf{setattr()} 및 \textbf{getattr()}
    \item \textbf{*args} 및 \textbf{**kwargs}
    \item \textbf{sys.\_getframe()}, \textbf{sys.exc\_info()}
  \end{itemize}
\end{slide}
}

\subsection{기타}
\frame{
\begin{slide}
\frametitle{PyPy의 약점 3 - 아직 최적화가 덜된 영역}
  다음 기능은 PyPy가 아직 충분히 성숙하지 못해 최적화가 덜된 영역으로, 앞으로 개선될 것이라고 합니다
  \begin{itemize}
    \item 매우 큰 \textbf{dict}: GC 구현 상의 문제로 느리다고 합니다
    \item \textbf{generator} 및 \textbf{itertools}
  \end{itemize}
\end{slide}
}

\section{PyPy의 미래}
\subsection{멀티스레드}
\frame{
\begin{slide}
\frametitle{PyPy의 미래 - 멀티스레딩}
  STM: Software Transactional Memory
  \begin{itemize}
    \item 모든 변수 접근이 Transaction으로 구현되어, 사용자가 명시적 lock을 걸 필요가 없는 구조
    \item 소프트웨어적으로 이를 구현하여 Global Interpreter Lock을 제거할 계획이라고 합니다
    \item 현재 알파 단계
    \item STM은 차세대 intel architecture에서 하드웨어적으로 구현될 예정
  \end{itemize}

  병렬 Garbage Collector
  \begin{itemize}
  \item 현재 GC는 싱글 스레드로 구현되어 있으며, 향후 멀티스레드 GC가 계획되어 있습니다
  \end{itemize}
\end{slide}
}

\subsection{CPU 지원}
\frame{
\begin{slide}
\frametitle{PyPy의 미래}
  다양한 CPU 지원
  \begin{itemize}
    \item 현재는 Intel CPU만 지원합니다
    \item ARM 및 PowerPC 지원이 계획되어 있습니다
  \end{itemize}
\end{slide}
}

\end{document}
  
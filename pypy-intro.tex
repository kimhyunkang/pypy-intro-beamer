\documentclass[slidestop]{beamer}
\usepackage{kotex}
\usepackage{hyperref}
\usepackage[accumulated]{beamerseminar}
\usepackage{beamertexpower}
\usepackage{beamerthemeGoettingen}
\usepackage{beamercolorthemebeaver}

\usepackage{tikz}

\usetikzlibrary{arrows,shapes}

\usepackage{graphicx}

\tikzstyle{format} = [draw, thin, fill=blue!20]
\tikzstyle{medium} = [ellipse, draw, thin, fill=green!20, minimum height=2.5em]

\addtobeamertemplate{block begin}{}{\setlength{\parskip}{35pt plus 1pt minus 1pt}}

\title{PyPy - A Python Interpreter written in Python}
\author[]{모비젠 플랫폼연구팀 김현강\\\code{kimhyunkang@mobigen.com}}
\date{}

\begin{document}

%% 표지
\begin{slide}

\maketitle

\end{slide}

\section{PyPy - 소개}
\frame{
\begin{slide}
\frametitle{PyPy}

  \includegraphics[width=0.8\textwidth]{Pypy_logo} \\
  \url{http://pypy.org}

  \begin{itemize}
  \item Python으로 작성한 Python 구현체
  \item CPython(표준 인터프리터)은 인터프리터가 모두 C로 작성되어 있습니다
  \item 반면 PyPy는 전부 Python으로 작성되어 있습니다
  \end{itemize}

\end{slide}
}

\frame{
\begin{slide}
\frametitle{PyPy의 역사}

  \begin{itemize}
  \item psyco의 개발자 Armin Rigo가 개발하였습니다
  \item 2007년 1.0 릴리즈가 발표되었습니다
  \item 현재 최신 릴리즈는 1.9이며, 2.0 베타가 발표되어 테스트 중에 있습니다
  \item Python 2.7.2 문법을 모두 지원합니다
    \begin{itemize}
    \item 표준 라이브러리도 모두 지원합니다
    \item Python 3은 아직 지원하지 않습니다
    \end{itemize}
  \item (Single-thread 환경에서) 평균적으로 가장 빠른 파이썬 구현체입니다
    \begin{itemize}
    \item Multi-thread 환경에서는 GIL이 없는 Jython이 더 빠를때도 있습니다
    \end{itemize}
  \item 현재 EU의 재정적 지원을 받아 개발중입니다
  \end{itemize}

\end{slide}
}

\section{Why PyPy?}
\frame{
\begin{slide}
\frametitle{PyPy에 관심을 갖게 된 이유}
  \begin{itemize}
  \item 빠릅니다
    \begin{itemize}
    \item 평균 5배 가량 성능이 향상된다고 합니다
    \end{itemize}
  \item 호환성을 중요하게 생각합니다
    \begin{itemize}
    \item 순수 파이썬 코드일 경우 거의 코드 변경 없이 PyPy를 도입할 수 있습니다
    \end{itemize}
  \item 소프트웨어를 어떻게 최적화해야 하는가에 관해 배울 수 있습니다
    \begin{itemize}
    \item 동적 언어로 만든 구현체를 정적 언어보다 빠르게 최적화하기 위해 도입한 여러 테크닉을 살펴볼 수 있습니다
    \end{itemize}
  \item 파이썬 언어 그 자체에 대해 배울 수 있습니다
    \begin{itemize}
    \item CPython과 PyPy의 비교를 통해 각 구현체의 내부에 대해 배울 수 있습니다
    \end{itemize}
  \end{itemize}
\end{slide}
}

\section{Bootstrapping}
\frame{
\begin{slide}
\frametitle{Bootstrapping}

  대상 언어로 대상 언어의 컴파일러/인터프리터를 짜는 것을 bootstrapping이라고 합니다
  \\~\\
  Bootstrapping은 컴파일러가 복잡할수록 유용한 테크닉입니다
  \begin{itemize}
  \item GCC (C)
  \item Microsoft Visual C++
  \item Sun JDK
  \item Perl 6
  \item GHC (Haskell)
  \end{itemize}
\end{slide}
}

\frame{
\begin{slide}
\frametitle{Bootstrapping}
  Bootstrapping 컴파일러의 장점
  \begin{itemize}
  \item 컴파일러를 잘 만들기 위해 다른 언어를 잘 알 필요가 없습니다
  \item 다른 아키텍처로 포팅하기가 쉽습니다
  \item 컴파일러가 최적화된 코드를 만들면 런타임뿐만 아니라 컴파일 속도가 함께 빨라집니다
    \begin{itemize}
    \item PyPy는 인터프리터이므로 해당되지 않습니다
    \end{itemize}
  \item 일반적으로 더 낮은 레벨의 언어로 컴파일러를 작성하게 되므로 컴파일러는 복잡해지기 쉽지만, bootstrapping 컴파일러는 대상 언어와 구현 언어가 같은 언어이므로 더 복잡한 기능을 쉽게 구현할 수 있습니다
  \end{itemize}
\end{slide}
}

\section{PyPy의 장점}
\frame{
\begin{slide}
\frametitle{PyPy의 장점}
  빠릅니다
  \begin{itemize}
  \item 프로그램의 종류에 따라 다르지만, 순수 파이썬으로 작성된 프로그램의 경우 CPython보다 3배 이상 빠릅니다
  \item \url{http://speed.pypy.org/}
  \end{itemize}

  메모리를 더 잘 관리합니다
  \begin{itemize}
  \item 자료구조에 따라 메모리를 50\% 가량 절약할 수 있습니다
  \item CPython보다 garbage collector 성능이 더 좋습니다
  \end{itemize}

  CPython과의 호환성을 철저히 관리합니다
  \begin{itemize}
  \item 홈페이지에 명시된 몇가지 코너케이스를 제외하면, CPython에서 돌아가는 코드는 PyPy에서도 돌아갑니다
  \end{itemize}
\end{slide}
}

\section{PyPy 성능의 비밀}
\frame{
\begin{slide}
\frametitle{PyPy 성능의 비밀}
  파이썬으로 짠 인터프리터가 어떻게 CPython보다 빠를 수 있나요?
  \begin{itemize}
  \item PyPy를 CPython으로 돌릴 수도 있습니다
    \begin{itemize}
    \item 당연하게도 느립니다 (CPython의 200배 정도)
    \end{itemize}
  \item PyPy는 Python의 부분집합인 RPython으로 만들어져 있습니다
  \end{itemize}
\end{slide}
}

\subsection{RPython}
\frame{
\begin{slide}
\frametitle{PyPy가 빠른 이유 1 - RPython}
  PyPy 인터프리터의 파이썬 소스코드는 RPython이라는 컴파일러를 통해 C로 번역됩니다
  \begin{itemize}
  \item 컴파일 시 각 변수의 타입을 분석하여 그에 해당하는 C 타입으로 번역됩니다
  \item 인터프리터의 소스코드는 Python이지만, 실제로는 C 언어를 거쳐 다시 어셈블리로 번역되므로 인터프리터의 실행속도는 빠릅니다
  \end{itemize}
  RPython
  \begin{itemize}
  \item PyPy를 제작하기 위해 만든 Python 언어의 부분집합
  \item 한 변수에 여러개의 타입을 가진 변수를 할당할 수 없습니다
  \item \textbf{os}, \textbf{sys} 등 일부 라이브러리만 사용할 수 있습니다
  \item RPython Compiler는 타입 추론 알고리즘을 거쳐 이 소스코드를 C 언어로 번역합니다
  \end{itemize}
\end{slide}
}

\frame{
\begin{slide}
\frametitle{PyPy가 빠른 이유 1 - RPython}

  \begin{figure}
  \begin{tikzpicture}[node distance=3cm, auto,>=latex', thick]
    \path[->] node[format] (pypy) {pypy source}
              node[format, below of=pypy, node distance=1.5cm] (pypyc) {translated C code}
              node[format, below of=pypyc, node distance=1.5cm] (runtime) {pypy runtime}
              (pypy) edge node {RPython Compiler} (pypyc)
              (pypyc) edge node {GCC} (runtime);
  \end{tikzpicture}
  \end{figure}
\end{slide}
}

\subsection{JIT Compilation}
\frame{
\begin{slide}
\frametitle{PyPy가 빠른 이유 2 - JIT Compilation}
  CPython은 코드 줄마다 여러가지 체크를 수행합니다
  \\~\\
  인터프리터가 \textbf{a+b} 를 실행할 때 일어나는 일
  \begin{itemize}
  \item 오브젝트 \textbf{a}가 \textbf{\_\_add\_\_} 연산자를 갖고 있는지 검사합니다
  \item 없을 경우 \textbf{b}가 \textbf{\_\_radd\_\_} 연산자를 갖고 있는지 검사합니다
  \item 둘 중에 하나가 있다면, 두 값의 타입을 검사하여 coercion을 수행합니다
  \item 실제 덧셈이 수행됩니다
  \end{itemize}
  이와같은 오버헤드가 쌓여 전체 성능을 저하시키게 됩니다
\end{slide}
}

\frame{
\begin{slide}
\frametitle{PyPy가 빠른 이유 2 - JIT Compilation}
  PyPy는 파이썬 코드를 실행할 때 일부 코드를 기계어로 실시간 번역합니다
  \begin{itemize}
  \item 이런 테크닉을 Just-in-time 컴파일이라고 합니다
  \item 모든 코드를 기계어로 번역하는 것은 실행시간보다 컴파일 시간이 오래 걸리므로 낭비입니다
  \item 루프 내의 코드나 자주 호출되는 함수가 나타나면 기계어로 번역합니다
  \end{itemize}
  JIT는 잘 알려진 테크닉이며, 여러 가상머신에서 사용되고 있습니다
  \begin{itemize}
  \item JVM
  \item .NET
  \item LLVM(Objective C, clang)
  \item V8(Google Chrome)
  \end{itemize}
\end{slide}
}

\frame{
\begin{slide}
\frametitle{PyPy가 빠른 이유 2 - Tracing JIT}
  PyPy의 JIT는 한단계 더 나아가, 실행중에 문법을 실시간 분석하여 사용되지 않는 조건분기 등을 기계어 코드에서 제거합니다
  \begin{itemize}
  \item 이런 테크닉을 Tracing JIT라고 합니다
  \item Mozilla Firefox의 Javascript Engine인 TraceMonkey에서도 유사한 테크닉이 사용됩니다
  \item CPU가 실시간으로 수행하고 있는 것과 비슷한 일이지만, PyPy는 소스코드를 분석하므로 최적화를 더 잘 수행할 수 있습니다
  \end{itemize}
\end{slide}
}

\subsection{Garbage Collector}
\frame{
\begin{slide}
\frametitle{PyPy가 빠른 이유 3 - Garbage Collector}
  CPython은 reference count 로 메모리를 관리합니다
  \begin{itemize}
  \item 모든 CPython 변수에는 몇개의 변수가 자기 자신을 가리키고 있는지 추적하는 값이 붙어 있으며, 이 숫자가 0이 될때 메모리에서 해제합니다
  \item 따라서, 모든 변수가 메모리를 4바이트씩 더 사용합니다
  \item 변수가 서로를 순환 참조할 경우 메모리가 해제되지 않습니다
    \begin{itemize}
    \item 이를 막기 위해 cycle detector를 따로 실행합니다
    \end{itemize}
  \item 또한, 변수 해제시마다 연결된 모든 메모리에 접근하여 count를 감소시켜야 하므로, cache 성능이 떨어집니다
  \item 병렬로 구현하기 어렵습니다
  \end{itemize}
\end{slide}
}

\frame{
\begin{slide}
\frametitle{PyPy가 빠른 이유 3 - Garbage Collector}
  PyPy는 주기적으로 한 변수가 다른 변수를 가리키는 포인터를 모두 추적한 후 쓰이지 않는 메모리를 모아 해제합니다
  \\~\\
  이런 테크닉을 Tracing Garbage Collection이라고 합니다
  \\~\\
  장점
  \begin{itemize}
  \item 실제 사용하는 메모리 외에 추가로 사용하는 메모리가 적습니다
  \item 서로 연결된 메모리를 추적하는 것이 아니라 메모리 주소 순서대로 추적하므로 CPU cache를 잘 활용합니다
  \item 메인 스레드와 독립적으로 병렬로 실행 가능합니다 (PyPy의 구현은 아직 메인 스레드에서 동작합니다)
  \end{itemize}
\end{slide}
}

\frame{
\begin{slide}
\frametitle{PyPy가 빠른 이유 3 - Garbage Collector}
  PyPy는 Generational Tracing GC를 사용합니다
  \begin{itemize}
  \item 오랫동안 해제되지 않는 변수들과 자주 할당되고 해제되는 변수를 구분하여 서로 다른 블록에 할당합니다
  \item 수명 주기가 오래된 변수는 주기적으로 더 오래된 블록들로 이동시킵니다
  \item 변수들을 수명 주기별로 모아 같은 블록으로 관리하며, 해제할때는 큰 블록을 한번에 OS에 반환하므로, 할당 및 해제의 오버헤드가 적습니다
  \end{itemize}

  잘 최적화된 Generational GC는 대부분의 경우 Reference Counting 보다 효율적이라는 것이 여러 가상머신에서 증명되어 있습니다
\end{slide}
}

\section{PyPy의 약점}
\subsection{C 모듈}
\frame{
\begin{slide}
\frametitle{PyPy의 약점 1 - 일부 C 모듈을 지원하지 않음}
  많은 Python 모듈이 CPython API를 이용해 작성되어 있습니다.
  \begin{itemize}
  \item PyPy는 내부 구현이 CPython과 다르므로, 이를 이용해 작성된 모듈은 PyPy에서는 사용할 수 없습니다.
  \item PyPy 2.0에서는 CPython 호환성 기능을 베타테스트중입니다. 그러나 완전히 호환되는 단계는 아니라고 합니다
  \item Python 2.7 표준 라이브러리는 모두 지원합니다
  \end{itemize}

\end{slide}
}

\subsection{JIT}
\frame{
\begin{slide}
\frametitle{PyPy의 약점 2 - JIT가 최적화하지 못하는 코드}
  PyPy의 성능은 대부분 JIT에서 오며, JIT가 최적화하지 못하는 코드의 경우 CPython보다 느립니다
  \begin{itemize}
  \item 모듈 코드의 대부분이 C에서 동작할 경우, CPython과 속도가 비슷합니다
    \begin{itemize}
    \item 예: \textbf{sqlite3}, \textbf{cProfile}, \textbf{cPickle}
    \end{itemize}
  \item 반복 사용되는 코드가 거의 없을 경우
    \begin{itemize}
    \item JIT는 여러번 사용되는 코드를 최적화하며, 이에 걸리는 시간을 JIT warm-up이라고 합니다
    \item 반복 사용되는 코드가 없으면 JIT가 warm-up할 시간이 없어 빨라지기 힘듭니다
    \item 0.2초 이하의 짧은 코드의 경우 CPython이 대체로 빠르다고 합니다
    \item 대부분의 코드가 한번만 실행되는 경우 (예: 단위 테스트) CPython이 빠릅니다
    \end{itemize}
  \end{itemize}
\end{slide}
}

\frame{
\begin{slide}
\frametitle{PyPy의 약점 2 - JIT가 최적화하지 못하는 코드}
  언어 구조를 dynamic하게 변경하는 코드는 기계어로 번역하기 힘들어 JIT가 최적화하지 못합니다
  \begin{itemize}
    \item \textbf{setattr()} 및 \textbf{getattr()}
    \item \textbf{*args} 및 \textbf{**kwargs}
    \item \textbf{sys.\_getframe()}, \textbf{sys.exc\_info()}
    \item \textbf{sys.exc\_info()}
      \begin{itemize}
      \item PyPy JIT에는 stack frame을 일부 제거하는 코드가 포함되어 있습니다
      \item \textbf{sys.exc\_info()} 함수는 호출 스택 정보를 반환해야 합니다
      \item 따라서, 이 함수를 사용하는 코드가 보이면 JIT는 주변 stack frame 최적화를 하지 않습니다
      \item \textbf{sys.exc\_info()}는 로깅 관련 함수에서 자주 사용됩니다
      \end{itemize}
  \end{itemize}
\end{slide}
}

\subsection{기타}
\frame{
\begin{slide}
\frametitle{PyPy의 약점 3 - 아직 최적화가 덜된 영역}
  다음 기능은 PyPy가 아직 충분히 성숙하지 못해 최적화가 덜된 영역으로, 앞으로 개선될 것이라고 합니다
  \begin{itemize}
    \item 매우 큰 \textbf{dict}
      \begin{itemize}
      \item GC 구현 상의 문제로 느리다고 합니다
      \end{itemize}
    \item \textbf{generator}
      \begin{itemize}
      \item 최근 버전에서 약간 개선되었습니다
      \end{itemize}
    \item \textbf{filter}, \textbf{reduce} 등 iterator 관련 함수들
      \begin{itemize}
      \item 이 함수들은 CPython에서도 \textbf{for} 루프에 비해 느립니다. PyPy에서도 마찬가지입니다.
      \end{itemize}
  \end{itemize}
\end{slide}
}

\section{PyPy의 미래}
\begin{frame}[fragile]
\begin{slide}
\frametitle{PyPy의 미래}
  Transactional Memory
  \begin{itemize}
    \item 모든 변수 접근이 Transaction으로 구현됨
    \item lock을 걸지 않고 낙관적으로 작업을 수행한 후, 충돌이 발견되면 되돌림
    \item 차세대 Intel architecture인 Haswell에서 하드웨어적으로 구현될 예정
  \end{itemize}
  PyPy-STM (Software Transactional Memory)
  \begin{itemize}
    \item 소프트웨어적으로 Transactional Memory를 구현하여 Global Interpreter Lock을 제거할 계획이라고 합니다
    \item 현재 구현중 - 아직 버그가 많다고 합니다
  \end{itemize}

%% \begin{verbatim}
%% with atomic:
%%     do something
%% \end{verbatim}
\end{slide}
\end{frame}

\frame{
\begin{slide}
\frametitle{PyPy의 미래}
  병렬 Garbage Collector
  \begin{itemize}
  \item 현재 GC는 싱글 스레드로 구현되어 있으며, 향후 멀티스레드 GC가 계획되어 있습니다
  \end{itemize}

  다양한 CPU 지원
  \begin{itemize}
    \item 현재는 Intel CPU만 지원합니다
    \item ARM 및 PowerPC 지원이 계획되어 있습니다
  \end{itemize}
\end{slide}
}

\end{document}
  
